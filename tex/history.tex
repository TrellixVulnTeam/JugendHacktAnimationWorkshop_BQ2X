\subsection{Geschichte der Animation}
\begin{frame}
    \frametitle{Frühe Animationen und optische Täuschung}
    \begin{itemize}
        \item Die Animation ist eine Technik, bei der durch die schnelle Abfolge von Einzelbildern der Eindruck von Bewegung entsteht.
        \item Das Prinzip heißt \href{https://de.wikipedia.org/wiki/Beta-Bewegung}{Beta-Bewegung}
    \end{itemize}
\end{frame}

\begin{frame}
    \frametitle{Frühe Animationen und optische Täuschung}
    \framesubtitle{Daumenkino}
    \begin{minipage}{0.5\textwidth}
        \begin{itemize}
            \item 1600
            \item Die erste in größerem Maßstab eingesetzte Technik war das \href{https://de.wikipedia.org/wiki/Daumenkino}{Daumenkino}.
            \item Bei dieser Technik wurden die Bilder auf einzelne Blätter in einem Buch geschrieben, das dann schnell durchgeblättert wurde.
        \end{itemize}
    \end{minipage} \hfill
    \begin{minipage}{0.45\textwidth}
        \grabto{https://upload.wikimedia.org/wikipedia/commons/1/1f/Linnet_kineograph_1886.jpg}{kineograph.jpg}
        \begin{figure}
            \includegraphics[width=\textwidth]{kineograph.jpg}
        \end{figure}
    \end{minipage}
\end{frame}

\begin{frame}
    \frametitle{Frühe Animationen und optische Täuschung}
    \framesubtitle{Laterna Magica}
    \begin{minipage}{0.5\textwidth}
        \begin{itemize}
            \item 1671
            \item Die \href{https://de.wikipedia.org/wiki/Laterna_magica}{Laterna Magica} ist ein Gerät zur Bildprojektion.
        \end{itemize}
    \end{minipage} \hfill
    \begin{minipage}{0.45\textwidth}
        \grabto{https://upload.wikimedia.org/wikipedia/commons/9/9d/Laterna_magica_Aulendorf.jpg}{laterna_magica.jpg}
        \begin{figure}
            \includegraphics[width=\textwidth]{laterna_magica.jpg}
        \end{figure}
    \end{minipage}
\end{frame}

\begin{frame}
    \frametitle{Frühe Animationen und optische Täuschung}
    \framesubtitle{Thaumatrop}
    \begin{minipage}{0.5\textwidth}
        \begin{itemize}
            \item 1825 \href{https://de.wikipedia.org/wiki/William_Henry_Fitton}{William Henry Fitton}
            \item \href{https://de.wikipedia.org/wiki/Thaumatrop}{Thaumatrop} ist eine flache Scheibe bei der auf jeder Seite ein Bild gezeichnet ist.
            \item Durch schnelle Rotation verschmelzen diese zu einem Bild.

        \end{itemize}
    \end{minipage} \hfill
    \begin{minipage}{0.45\textwidth}
        \grabto{https://upload.wikimedia.org/wikipedia/commons/9/9f/Taumatropio_fiori_e_vaso\%2C_1825.gif}{taumatrop.gif}
        \convertGif{taumatrop.gif}{taumatrop.mp4}
        \includemedia[
        activate=pageopen,
        width=\textwidth,height=\textwidth,
        addresource=taumatrop.mp4,
        flashvars={%
        src=taumatrop.mp4      % same path as in addresource!
        &autoPlay=true      % optional configuration
        &loop=true          % variables
        }
        ]{}{StrobeMediaPlayback.swf}
    \end{minipage}
\end{frame}

\begin{frame}
    \frametitle{Stroboskopische Animation}
    \begin{itemize}
        \item Die Stroboskopische Animation oder \href{https://de.wikipedia.org/wiki/Beta-Bewegung}{Stroboskopische Bewegung} basiert auf der schnellen Abfolge von Bildern, bei der zwischen den Bildern immer eine Dunkelphase eingebaut wird.
        \item Das führt dann zu der \href{https://de.wikipedia.org/wiki/Persistenz_des_Sehens\#Flimmerfusion}{Flimmerfusion}, bei der die verschiedenen Einzelbilder zu einer kontinuierlichen Bewegung verschmelzen.
        \item Das ist die Basis auf der alle heutigen Animationstechniken basieren.
    \end{itemize}
\end{frame}

\begin{frame}
    \frametitle{Stroboskopische Animation}
    \framesubtitle{Phenakistiskop}
    \begin{minipage}{0.5\textwidth}
        \begin{itemize}
            \item 1833 \href{https://de.wikipedia.org/wiki/Joseph_Antoine_Ferdinand_Plateau}{Joseph Plateau}, \href{https://de.wikipedia.org/wiki/Simon_Stampfer}{Simon Stampfer}
            \item \href{https://de.wikipedia.org/wiki/Phenakistiskop}{Phenakistiskop} war die erste Technik um Bewegungsabläufe in Form von flüssigen Animationen darzustellen.
        \end{itemize}
    \end{minipage} \hfill
    \begin{minipage}{0.45\textwidth}
        \grabto{https://upload.wikimedia.org/wikipedia/commons/thumb/8/8a/Phenakistoscope_3g07690u.jpg/1024px-Phenakistoscope_3g07690u.jpg}{phenakistiskop.jpg}

        \includegraphics[width=\textwidth]{phenakistiskop.jpg}
    \end{minipage}
\end{frame}

\begin{frame}
    \frametitle{Stroboskopische Animation}
    \framesubtitle{Phenakistiskop}
    \begin{minipage}{0.5\textwidth}
        \begin{itemize}
            \item Man blickt durch Löcher auf der Rückseite der Scheibe auf einen Spiegel. Wenn man die Scheibe in Drehung versetzt entsteht ein stroboskopischer Effekt, der die Einzelbilder immer an verschiedenen Positionen erscheinen lässt und so zur Verschmelzung der Einzelbilder zu einer kontinuierlichen Bewegung führt.
        \end{itemize}
    \end{minipage} \hfill
    \begin{minipage}{0.45\textwidth}
        \grabto{https://upload.wikimedia.org/wikipedia/commons/d/d3/Phenakistoscope_3g07690d.gif}{phenakistiskop.gif}

        \convertGif{phenakistiskop.gif}{phenakistiskop.mp4}
        \includemedia[
        activate=pageopen,
        width=\textwidth,height=\textwidth,
        addresource=phenakistiskop.mp4,
        flashvars={%
        src=phenakistiskop.mp4      % same path as in addresource!
        &autoPlay=true      % optional configuration
        &loop=true          % variables
        }
        ]{}{StrobeMediaPlayback.swf}
    \end{minipage}
\end{frame}

\begin{frame}
    \frametitle{Stroboskopische Animation}
    \framesubtitle{Zoetrop}
    \begin{minipage}{0.5\textwidth}
        \begin{itemize}
            \item 1834 \href{https://de.wikipedia.org/wiki/William_George_Horner}{William George Horner}
            \item \href{https://de.wikipedia.org/wiki/Zoetrop}{Zoetrop} gleiches Prinzip wie das Phenakistiskop.
            \item Abgewandelter Aufbau um Zeichnungen nebeneinander auf einem einzelnen Blatt Papier anzugertigen.
            \item Wie Heute in der Computergrafik und bei Spielen bei der \href{https://de.wikipedia.org/wiki/Sprite_(Computergrafik)\#Spriteanimation}{Sprite Animation}.
        \end{itemize}
    \end{minipage} \hfill
    \begin{minipage}{0.45\textwidth}
        \grabto{https://upload.wikimedia.org/wikipedia/commons/9/99/Zoetrope.jpg}{zoetrope.jpg}
        \includegraphics[width=\textwidth]{zoetrope.jpg}
    \end{minipage}
\end{frame}

\begin{frame}
    \frametitle{Stroboskopische Animation}
    \framesubtitle{Fotografie}
    \begin{minipage}{0.5\textwidth}
        \begin{itemize}
            \item 1830er Jahre
            \item \href{https://de.wikipedia.org/wiki/Fotografie}{Fotografie} zuerst mussten die Bilder noch lange Zeit belichtet werden und es durfte sich nichts bewegen.
            \item Es dauerte noch lange bis an Aufnahmen von Bewegungen oder Film überhaupt zu denken war.
        \end{itemize}
    \end{minipage} \hfill
    \begin{minipage}{0.45\textwidth}
        \grabto{https://upload.wikimedia.org/wikipedia/commons/c/c5/Photographer1850s.png}{fotografie.jpg}
        \includegraphics[width=\textwidth]{fotografie.jpg}
    \end{minipage}
\end{frame}

\begin{frame}
    \frametitle{Stroboskopische Animation}
    \framesubtitle{Mutoskop}
    \begin{minipage}{0.5\textwidth}
        \begin{itemize}
            \item Ab 1860
            \item \href{https://de.wikipedia.org/wiki/Mutoskop}{Mutoskop} mehrere Einzelbilder auf eine Scheibe aufgebracht, die gedreht wird.
            \item Durch einen Anschleg werden diese kurz angehalten und können so betrachtet weden.
        \end{itemize}
    \end{minipage} \hfill
    \begin{minipage}{0.45\textwidth}
        \grabto{https://upload.wikimedia.org/wikipedia/commons/3/33/Mutoscope_1896.jpg}{mutoscop.jpg}
        \includegraphics[width=\textwidth]{mutoscop.jpg}
    \end{minipage}
\end{frame}

\begin{frame}
    \frametitle{Stroboskopische Animation}
    \framesubtitle{Mutoskop}
    \begin{minipage}{0.5\textwidth}
        \begin{itemize}
            \item 1877 \href{https://de.wikipedia.org/wiki/\%C3\%89mile_Reynaud}{Emile Reynaud}
            \item Sozusagen eine Kombination aus \href{https://de.wikipedia.org/wiki/Laterna_magica}{Laterna Magica} und \href{https://de.wikipedia.org/wiki/Zoetrop}{Zoetrop}.
            \item Ein Vorläufer der \href{https://de.wikipedia.org/wiki/Bewegte_Bilder}{Kinematographie}.
        \end{itemize}
    \end{minipage} \hfill
    \begin{minipage}{0.45\textwidth}
        \grabto{https://upload.wikimedia.org/wikipedia/commons/6/60/Lanature1882_praxinoscope_projection_reynaud.png}{praxinoscop.png}
        \includegraphics[width=\textwidth]{praxinoscop.png}
    \end{minipage}
\end{frame}

%### Chronofotografie
%
%Zwischen 1872 und 1879 Begann [Eadweard Muybridge](https://de.wikipedia.org/wiki/Eadweard_Muybridge) mit Experimententen zu Fotografie und Bewegung.
%
%Er begann auf Grund einer Auftragsarbeit damit die Bewegung von Pferden im Galopp zu analysieren und erstellte daraus eine Serienfotographie.
%
%![Horse in motion](https://upload.wikimedia.org/wikipedia/commons/thumb/d/d2/The_Horse_in_Motion_high_res.jpg/1280px-The_Horse_in_Motion_high_res.jpg)
%
%![Horse in motion animated](https://upload.wikimedia.org/wikipedia/commons/d/dd/Muybridge_race_horse_animated.gif)
%
%In den Folgenden Jahren Arbeitete er an weiteren Analysen von Bewegungen anhand schnell nach einander aufgenommener Fotografien.


\begin{frame}
    \frametitle{Stroboskopische Animation}
    \framesubtitle{Chronofotografie}
    \begin{minipage}{0.5\textwidth}
        \begin{itemize}
            \item Zwischen 1872 und 1879 \href{https://de.wikipedia.org/wiki/Eadweard_Muybridge}{Eadweard Muybridge}
            \item Experimentente zu Fotografie und Bewegung.
            \item Auftragsarbeit um die Bewegung von Pferden im Galopp zu analysieren
            \item Erstellte daraus eine Serienfotographie.

        \end{itemize}
    \end{minipage} \hfill
    \begin{minipage}{0.45\textwidth}
        \grabto{https://upload.wikimedia.org/wikipedia/commons/thumb/d/d2/The_Horse_in_Motion_high_res.jpg/1280px-The_Horse_in_Motion_high_res.jpg}{horse.jpg}
        \includegraphics[width=\textwidth]{horse.jpg}
    \end{minipage}
\end{frame}

\begin{frame}
    \frametitle{Stroboskopische Animation}
    \framesubtitle{Chronofotografie}
    \begin{minipage}{0.5\textwidth}
        \begin{itemize}
            \item In den Folgenden Jahren Arbeitete er an weiteren Analysen von Bewegungen anhand schnell nach einander aufgenommener Fotografien.
        \end{itemize}
    \end{minipage} \hfill
    \begin{minipage}{0.45\textwidth}
        \grabto{https://upload.wikimedia.org/wikipedia/commons/d/dd/Muybridge_race_horse_animated.gif}{horseanimation.gif}
        \convertGif{horseanimation.gif}{horseanimation.mp4}
        \includemedia[
        activate=pageopen,
        width=\textwidth,height=\textwidth,
        addresource=horseanimation.mp4,
        flashvars={%
        src=horseanimation.mp4      % same path as in addresource!
        &autoPlay=true      % optional configuration
        &loop=true          % variables
        }
        ]{}{StrobeMediaPlayback.swf}
    \end{minipage}
\end{frame}

\begin{frame}
    \frametitle{Stroboskopische Animation}
    \framesubtitle{Zoopraxiskop}
    \begin{minipage}{0.5\textwidth}
        \begin{itemize}
            \item 1879 \href{https://de.wikipedia.org/wiki/Eadweard_Muybridge}{Eadweard Muybridge}
            \item \href{https://de.wikipedia.org/wiki/Zoopraxiskop}{Zoopraxiskop} eine Möglichkeit chronofotografischen Aufnahmen mit einer Glühlampe auf eine Oberfläche zu projizieren 
        \end{itemize}
    \end{minipage} \hfill
    \begin{minipage}{0.45\textwidth}
        \grabto{https://upload.wikimedia.org/wikipedia/commons/6/6b/Zoopraxiscope_16485u.gif}{zoopraxiscope.gif}
        \convertGif{zoopraxiscope.gif}{zoopraxiscope.mp4}
        \includemedia[
        activate=pageopen,
        width=\textwidth,height=\textwidth,
        addresource=zoopraxiscope.mp4,
        flashvars={%
        src=zoopraxiscope.mp4      % same path as in addresource!
        &autoPlay=true      % optional configuration
        &loop=true          % variables
        }
        ]{}{StrobeMediaPlayback.swf}
    \end{minipage}
\end{frame}

\begin{frame}
    \frametitle{Stroboskopische Animation}
    \framesubtitle{Kaiserpanorama}
    \begin{minipage}{0.5\textwidth}
        \begin{itemize}
            \item 1880 von August Fuhrmann
            \item \href{https://de.wikipedia.org/wiki/Kaiserpanorama}{Kaiserpanorama} war die erste massentaugliche Möglichkeit \href{https://de.wikipedia.org/wiki/Stereoskopie}{stereoskopische} Animationen zu betrachten.
            \item Erlaubt es den Eindruck einer dreidimensionalen Animation wahrzunehmen.
        \end{itemize}
    \end{minipage} \hfill
    \begin{minipage}{0.45\textwidth}
        \grabto{https://upload.wikimedia.org/wikipedia/commons/2/20/August_Fuhrmann-Kaiserpanorama_1880.jpg}{kaiser.jpg}
        \includegraphics[width=\textwidth]{kaiser.jpg}
    \end{minipage}
\end{frame}

\begin{frame}
    \frametitle{Stroboskopische Animation}
    \framesubtitle{Elektrotachyscop}
    \begin{minipage}{0.5\textwidth}
        \begin{itemize}
            \item 1886 \href{https://de.wikipedia.org/wiki/Ottomar_Ansch\%C3\%BCtz}{Ottomar Anschütz}
            \item \href{https://de.wikipedia.org/wiki/Elektrotachyscop}{Elektrotachyscop} projiziert chronofotografische Aufnahmen über eine stroboskopische Scheibe.
        \end{itemize}
    \end{minipage} \hfill
    \begin{minipage}{0.45\textwidth}
        \grabto{https://upload.wikimedia.org/wikipedia/commons/e/e8/Electrotachyscope1.jpg}{electrotachyscope.jpg}
        \includegraphics[width=\textwidth]{electrotachyscope.jpg}
    \end{minipage}
\end{frame}

\begin{frame}
    \frametitle{Stroboskopische Animation}
    \framesubtitle{Kinetoskop}
    \begin{minipage}{0.5\textwidth}
        \begin{itemize}
            \item 1891 \href{https://de.wikipedia.org/wiki/William_Kennedy_Laurie_Dickson}{William Kennedy Laurie Dickson}
            \item \href{https://de.wikipedia.org/wiki/Kinetoskop}{Kinetoskop} ist der erste \href{https://de.wikipedia.org/wiki/Filmbetrachter}{Filmbetrachter}.
            \item Schaut auf einen Endlosfilm von durchscheinenden Filmpositiven die durch eine Blende für kurze Zeit von eine Glühlampe belichtet wurden.
        \end{itemize}
    \end{minipage} \hfill
    \begin{minipage}{0.45\textwidth}
        \grabto{https://upload.wikimedia.org/wikipedia/commons/0/0b/Kinetoscope.jpg}{kinetoscope.jpg}
        \includegraphics[width=\textwidth]{kinetoscope.jpg}
    \end{minipage}
\end{frame}


\begin{frame}
    \frametitle{D}
    \begin{itemize}
        \item D
    \end{itemize}

    %\includegraphics[width=.45\textwidth]{Linnet_kineograph_1886.jpg}
\end{frame}
