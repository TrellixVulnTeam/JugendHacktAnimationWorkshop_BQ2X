\subsection{Wahrnehmung}

\begin{frame}
    \frametitle{Wahrnehmung}
    \begin{itemize}
        \item Wenn wir über Animationen reden, reden wir zwangsweise auch über die menschliche Wahrnehmung.
        \item Tatsächlich reden wir hier über die Grenzen der menschlichen Wahrnehmung.
        \item Wann unser Auge und unser Gehirn eine Abfolge von einzelnen Bildern als eine kontinuierliche Bewegung wahrnimmt.
        \item wie es in der \href{https://de.wikipedia.org/wiki/Beta-Bewegung}{Beta-Bewegung} beschrieben wird.
    \end{itemize}
\end{frame}

\begin{frame}
    \frametitle{Wahrnehmung}
    \framesubtitle{Auge}
    \begin{itemize}
        \item Unser Auge ist das Organ mit dem wir sehen.
        \item Wenn wir Animationen verstehen wollen, müssen wir auch verstehen wie unser Auge funktioniert.
    \end{itemize}
\end{frame}

\begin{frame}
    \frametitle{Wahrnehmung}
    \framesubtitle{Aufbau}
    \begin{minipage}{0.5\textwidth}
        \begin{itemize}
            \item Stark vereinfacht besteht unser \href{https://de.wikipedia.org/wiki/Auge\#Augapfel}{Auge} aus zwei Teilen.
            \item Einer \href{https://de.wikipedia.org/wiki/Linse_(Auge)}{Linse}, die das eintretende Licht auf unsere
            \item \href{https://de.wikipedia.org/wiki/Netzhaut}{Netzhaut} projiziert.
        \end{itemize}
    \end{minipage} \hfill
    \begin{minipage}{0.45\textwidth}
        \grabto{https://upload.wikimedia.org/wikipedia/commons/thumb/a/a5/Eye_scheme.svg/1280px-Eye_scheme.svg.png}{auge.png}
        \begin{figure}
            \includegraphics[width=\textwidth]{auge.png}
        \end{figure}
    \end{minipage}
\end{frame}

\begin{frame}
    \frametitle{Wahrnehmung}
    \framesubtitle{Netzhaut}
        \begin{itemize}
            \item Auf unserer \href{https://de.wikipedia.org/wiki/Netzhaut}{Netzhaut} gibt es verschiedene Zellen die auf das eintreffende Licht reagieren.
            \item \href{https://de.wikipedia.org/wiki/St\%C3\%A4bchen_(Auge)}{Stäbchen} können Helligkeitsunterschiede wahrnehmen und sind sehr sensibel.
            \item \href{https://de.wikipedia.org/wiki/Zapfen_(Auge)}{Zapfen} dienen der Farbwahrnehmung. Wir Manschen haben drei verschiedene, jeweils für Rot, Grün und Blau, sie sind jedoch weniger sensibel als die Stäbchen.
            \item Wir haben etwa 120 Mio Stäbchen und 6 Mio Zapfen in unseren Augen.
        \end{itemize}
\end{frame}

\begin{frame}
    \frametitle{Wahrnehmung}
    \framesubtitle{Farben}
        \begin{itemize}
            \item Wie wir vom Auge schon gelernt haben können wir drei verschiedene Farben wahrnehmen.
            \item Das sind Rot, Grün und Blau.
            \item Mischt man diese Farben in verschiedenen Anteilen können wir alle möglichen Farben erzeugen.
            \item Dazu gibt es zwei Möglichkeiten.
        \end{itemize}
\end{frame}

\begin{frame}
    \frametitle{Wahrnehmung}
    \framesubtitle{Subtraktive Farbmischung}
    \begin{minipage}{0.5\textwidth}
        \begin{itemize}
            \item Die \href{https://de.wikipedia.org/wiki/Subtraktive_Farbmischung}{Subtraktive Farbmischung} ist das was wir aus dem Kunstunterricht in der Schule kennen.
            \item Wir haben ein weisses Blatt Papier, dass alle Farben reflektiert.
            \item Tragen wir auf das weisse Blatt Farben auf entfernen wir bestimmte Farbreize vom reflektierten Licht durch Filter oder Absorbtion.
        \end{itemize}
    \end{minipage} \hfill
    \begin{minipage}{0.45\textwidth}
        \grabto{https://upload.wikimedia.org/wikipedia/commons/thumb/6/64/CMY_ideal_version_rotated.svg/1024px-CMY_ideal_version_rotated.svg.png}{cmy.png}
        \begin{figure}
            \includegraphics[width=\textwidth]{cmy.png}
        \end{figure}
    \end{minipage}
\end{frame}

\begin{frame}
    \frametitle{Wahrnehmung}
    \framesubtitle{Subtraktive Farbmischung}
    \begin{minipage}{0.5\textwidth}
        \begin{itemize}
            \item Die Grundfarben sind:
            \item Cyan
            \item Magenta
            \item Gelb
        \end{itemize}
    \end{minipage} \hfill
    \begin{minipage}{0.45\textwidth}
        \grabto{https://upload.wikimedia.org/wikipedia/commons/thumb/6/64/CMY_ideal_version_rotated.svg/1024px-CMY_ideal_version_rotated.svg.png}{cmy.png}
        \begin{figure}
            \includegraphics[width=\textwidth]{cmy.png}
        \end{figure}
    \end{minipage}
\end{frame}

\begin{frame}
    \frametitle{Wahrnehmung}
    \framesubtitle{Additive Farbmischung}
    \begin{minipage}{0.5\textwidth}
        \begin{itemize}
            \item Bei der \href{https://de.wikipedia.org/wiki/Additive_Farbmischung}{Additiven Farbmischung} werden verschiedene Farbreize kombiniert und damit andere Farben erzeugt.
        \end{itemize}
    \end{minipage} \hfill
    \begin{minipage}{0.45\textwidth}
        \grabto{https://upload.wikimedia.org/wikipedia/commons/thumb/e/e0/Synthese\%2B.svg/1024px-Synthese\%2B.svg.png}{rgb.png}
        \begin{figure}
            \includegraphics[width=\textwidth]{rgb.png}
        \end{figure}
    \end{minipage}
\end{frame}

\begin{frame}
    \frametitle{Wahrnehmung}
    \framesubtitle{Additive Farbmischung}
    \begin{minipage}{0.5\textwidth}
        \begin{itemize}
            \item Die Grundfarben sind:
            \item Rot
            \item Grün
            \item Blau
        \end{itemize}
    \end{minipage} \hfill
    \begin{minipage}{0.45\textwidth}
        \grabto{https://upload.wikimedia.org/wikipedia/commons/thumb/e/e0/Synthese\%2B.svg/1024px-Synthese\%2B.svg.png}{rgb.png}
        \begin{figure}
            \includegraphics[width=\textwidth]{rgb.png}
        \end{figure}
    \end{minipage}
\end{frame}

